%%%%%%%%%%%%%%%%%%%%%%%%%%%%%%%%%%%%%%%%%%%%%%%%%%%%%%%%%%%%%%%%%%%%%
%% Title: SOP LaTeX Template
%% Author: Soonho Kong / soonhok@cs.cmu.edu
%% Created: 2012-11-12
%%%%%%%%%%%%%%%%%%%%%%%%%%%%%%%%%%%%%%%%%%%%%%%%%%%%%%%%%%%%%%%%%%%%%

%%%%%%%%%%%%%%%%%%%%%%%%%%%%%%%%%%%%%%%%%%%%%%%%%%%%%%%%%%%%%%%%%%%%%
%%
%% Requirement:
%%     You need to have the `Adobe Caslon Pro` font family.
%%     For more information, please visit:
%%     http://store1.adobe.com/cfusion/store/html/index.cfm?store=OLS-US&event=displayFontPackage&code=1712
%%
%% How to Compile:
%%     $ xelatex main.tex
%%
%%%%%%%%%%%%%%%%%%%%%%%%%%%%%%%%%%%%%%%%%%%%%%%%%%%%%%%%%%%%%%%%%%%%%

\documentclass[letterpaper, 12pt ]{article}
\usepackage[letterpaper,margin=1 in,noheadfoot]{geometry}
\usepackage{fontspec, color, enumerate, sectsty}
\setlength{\footskip}{20pt}
\sectionfont{\large}
\subsectionfont{\small}


\usepackage[normalem]{ulem}
%


%\usepackage{titlesec}
%\titleformat*{\section}{\fontsize{18}{20}\selectfont}
%\titleformat*{\subsection}{\fontsize{13}{17}\selectfont}
%%%%%%%%%%%%%%%%%%%%%%%%%%%%%%%%%%%%%%%%%%%%%%%%%%%%%%%%%%%%%%%%%%%%%
%                      YOUR INFORMATION
%
%      PLEASE EDIT THE FOLLOWING LINES ACCORDINGLY!!
%%%%%%%%%%%%%%%%%%%%%%%%%%%%%%%%%%%%%%%%%%%%%%%%%%%%%%%%%%%%%%%%%%%%%
\newcommand{\soptitle}{Fulbright Scholar Research Award Project Statement}
\newcommand{\yourname}{Alejandra Sanchez-Rios}
\newcommand{\youremail}{asanchez@ceoas.oregonstate.edu}

\usepackage[bookmarks, colorlinks, breaklinks,
pdftitle={\yourname - \soptitle},pdfauthor={\yourname}, unicode]{hyperref}
\hypersetup{linkcolor=magneta,citecolor=magenta,filecolor=magenta,urlcolor=[named]{WildStrawberry}}

\usepackage{setspace}
\pagestyle{plain}
\singlespacing
%%%%%%%%%%%%%%%%%%%%%%%%%%%%%%%%%%%%%%%%%%%%%%%%%%%%%%%%%%%%%%%%%%%%%
%                      Title and Author Name
%%%%%%%%%%%%%%%%%%%%%%%%%%%%%%%%%%%%%%%%%%%%%%%%%%%%%%%%%%%%%%%%%%%%%
\begin{document}
\begin{center}{\Large \scshape \soptitle}\end{center}
\begin{center}\vspace{0.2em} { \yourname\\}
\end{center}

%%%%%%%%%%%%%%%%%%%%%%%%%%%%%%%%%%%%%%%%%%%%%%%%%%%%%%%%%%%%%%%%%%%%%
%                      SOP Body
% NOTE: Use \amper instead of \&
%%%%%%%%%%%%%%%%%%%%%%%%%%%%%%%%%%%%%%%%%%%%%%%%%%%%%%%%%%%%%%%%%%%%%

Collaborative work has become an intrinsic strategy in the field of Physical Oceanography, and it has enabled us to study complex processes in different regions of the ocean that were not possible before. For instance, to analyze where mixing of different types of water mass occurs at small-scales (<10 km), field observations rely on constant surveillance with expensive and specialized equipment, e.g., microstructure sensors that measure turbulence. This economical and technical burden can be shared between different institutes and nations, bringing together sufficient equipment to create a successful campaign tailored to the nature of turbulence, a physical process that is highly intermittent and small in spatial and temporal scales. Projects also benefit from the collaboration between interdisciplinary scholars. A team composed of ocean modelers, observational oceanographers, physicists, atmospheric scientists, and engineers, can tackle a problem from different angles, advancing the field further and exploring new scientific questions. International collaboration creates an opportunity to study different regions of the ocean with similar processes, and creates a inclusive community that address global issues. As a PhD student at Oregon State University (OSU), I have been fortunate to work along side different institutes and different countries, specially in Taiwan, and I have come to understand how invaluable working in such a team is. As I transition to a new phase in my career, I want to continue building and strengthening connections with the Taiwanese scientific community, and The Fulbright scholar award would pave the way for me to collaborate with the Institute of Oceanography in the National Taiwan University (IONTU), but most importantly, will allow me to be part of a scientific community that I have only known as an external collaborator. The project I propose to carry out with Professor Shih-Nan Chen is the next step from my dissertation work where I plan to investigate and describe mechanisms that facilitate mixing of two different water masses at the edge of a front formed by the Kuroshio Current, a major source of nutrients, heat, and salt to the region. Through the collaboration with IONTU, I will have access to the extensive field data sets from the region and benefit from the faculty experience in numerical models of processes I'm studying. Moreover, as a visiting scholar I can share my expertise in data analysis of small scale features, and my field work experience with underwater vehicles of different types with the Taiwanese scientific community and IONTU. 


% I studied the Kuroshio Current, which  impacts atmospheric and ecological systems in Taiwan. 
\section*{Background \&  Research Experience}

    \paragraph{}
    Before introducing the details and objectives of the proposed project, I want to describe the processes that interest me and comment about current studies in the field that have lead to development of new questions and new challenges
    
\subsection*{The Importance of Ocean Fronts}

\paragraph{}
    Ocean fronts are regions that bring together water masses of different characteristics. These differences creates gradients in temperature and salinity which are of particular interest due to the way both properties inversely affect density of seawater. Stable lateral gradients in density maintain ocean currents, and are reservoirs of potential energy. Observations and model outputs have shown how this equilibrium can be easily disturb by a wide type of factors: strong wind can injects momentum into the upper layers, heat exchange with the atmosphere will change the density of the water mass, temporal variability of the location of the front will change the density gradient, and differences in velocities at different depth will facilitate vertical shear, to name a couple of them. All these physical conditions facilitate processes that will enhance mixing between the two different water masses and are important because they enable exchange of heat, nutrients, inorganic and organic matter across different regions in the ocean and between the upper layers with deeper ocean. 
    
    \paragraph{}
    The processes involve with mixing in frontal systems are diverse, a small list can include: instabilities in the mixed layer, buoyancy convection, frontogenesis, subduction of water mass, lateral and vertical shear, and many others. These processes often results in stirring at the edge of the front, forming small-scale features, e.g interleavings, filaments and eddies, in which mixing mechanisms can act upon faster. The study of these features has benefited by the advances in technology that permits fast and continues measurements and the increase capabilities of computers to resolve high-resolution models. Research is being done to quantify the exchange of heat and salt in frontal regions and understand which processes are the main players in this transport. Past studies have demonstrated that the interaction between the different process is also equally important and.  The study of mixing and how water density is changed, known as water transformation, is crucial for modeling how currents transport tracers across ocean basis, and to investigate how this transport will be affected by the global circulation and atmospheric changes. Important ocean currents that transport great quantities of heat and salt across ocean basins are western boundary currents. I have focus my research in two of them: the Gulf Stream and the Kuroshio Current.
    
    
    % The air-sea interaction is also important to understand
    %  The upper ocean interacts with the atmosphere, sequestering CO2, exchanging heat, and acquiring O2. Water in the surface layers can then be transported into deeper depths, and deep water, rich in nutrients can be brought up to the surface where it can provide light and oxygen to the organism living in it. As its apeprent there are lot of processes that take place in ocean fronts, the complexity of the conectinos of these preocesses and ther sum impact is one fo the new frontie in oceangoraphy, In an effor toknow more about the fronts evulions, I have participated in crusies that have expliret he variability of thises proceses in kuroshio curent. This has inspired e to  know in detailed the processes that we have observed in the hight-resolution data obtained from the seagliders. 

\subsection*{The Kuroshio Current}
\begin{enumerate}
    \item  One of the main reasons I want to carry out a research project in Taiwan, is the direct impact the Kuroshio current has over this region. The KC impacts storms tracks, regional climate and the marine ecosystem. 
    
    \item Description of the Kuroshio Current and the intrusions into South China Sea
    
    \item Description of studies done east of the Kuroshio current and in the South China Sea
    
    % This currents is born when the west bound equatorian current meet the Philippians and splits in two, the MC which travels south, and the Kuroshio current that travels north. Once in its way towards the north, the kuroshio current will continue and pass east of Taiwan and and hit the East China sea, where it will then be deflectes towards the east. The Kuroshio current can also intrude thrught the luzon straing towards the South China Sea, concting this basin with the Pacific ocean. The nature of this intrusin is vastly study, and the proceses that involve the mixing of the water transported by th eKC and the SCS water was the focus of the second chapter of my dissertaion. 
    
    \end{enumerate}
    
    \subsection*{Previous research}  
    
  \begin{enumerate}
  
    \item For my dissertation I have identify and characterized structures along ocean fronts that facilitate the lateral exchange of heat and salt across both currents: the Gulf Stream and the Kuroshio Current. For both location I used data collected from gliders, microstructure sensors and fast profiling conductivity, depth and salinity profilers, to identify
    
    % I have extensive experience with oceanographic instruments that measure ocean properties needed to understand the dynamics occurring in the front, specially autonomous underwater vehicles, commonly knows as Gliders. I have been trained in deploying, piloting, and recovering them
    
    \item For the collection of data of these both studies, I was involve in research cruises extensively, and collaborated with faculty from different institutes in the US and Taiwan. In both cases we used satellite data to follow the location of the fronts, once identify, with deployed several gliders and profile the region. We carry out three cruises south of Taiwan in collaboration with faculty from IONTU. During this field work, I was trained in the deployment and recovery of oceanographic instruments, and was part the glider team that pilot them and interpret the data.  I invested a lot time in the anlisys of the extensive and complex data set.
    
    \item Using high-resolution \textit{in situ} measurements and satellite data, we created a 3D image (30 km across, 600 km along and 200m deep) of the Gulf Stream front over four days and evaluated the change in time of both heat and salt in the upper layer. We identified several small-scale features (10-50 km) of warm water being carry across the front and observed there was an increase of temperature and salinity associated to the location of the front. We concluded that the only possible source for the change, was lateral exchange, and the estimated rate was similar to past studies estimates. 
    
    \item I also investigated  water transformation at the edge of the Kuroshio Current as it intrudes to the South China Sea. We identify temperature and salinity interleaving throught the water column at the edge of the KC, and explore the mechanism  responsible for mixing, double diffusion or shear turbulence. Both found that both mechanism were present in the data set. This project is part of of Office of Naval Research (ONR) partnership with IONTU, were several US institutes participate. 
    % maybe mention in this part how this partnership allowed me to learn the research projects carry out in IONTU, and have a glimpse of the academic cultural 
    
% and correlated this features with the observed changes.  labeled streamers. Streamers are warm salty water  from the Gulf stream interleaved with the cold-fresh water of the Slop Water from the Mid-Atlantic bit. We concluded that thes
    % Observting the complexity and the diversty of procesies in has inspired me to dig deeper in understanding them. And Interacting with several resereacher from the IONTU in conferences and report meetings I have learned the resources and the collective exprienece these institution has about this area and experience owrking with numerical models which help them study with more detail thier thoeries and
    
\end{enumerate}

\section*{Project Overview}

Observing data from the KC front, I was able to observed a huge range of density, temperature and salinity structures and patterns, which has inspired me to investigate further how these patters are form in the front, what process are more important? How do they relate to regional circulation? How common are in the region? What triggers them? and dive deeper into the physical theory behind them. During my time in IONTU, I propose to do this by using simple numerical models to investigate different mechanism responsible for mixing at ocean fronts that enables mixing and variability along the front, and comparing them to real cases. 

 \subsection*{Objectives \& Methodology}
The main objective would be to explore the effect of wind forcing, instabilities, and stratification in the density structure of a front similar to the KC front, and investigate how this effects lead to water mass transformation. As well, to explore the type of patterns develop in the front when these mechanism are present and compare it to past observations of the KC. Using of idealized numerical simulations, will model realistic scenarios known to destabilized fronts,  e.g., downfront wind, baroclinic instability, to analyze the effect in the density and velocity three-dimensional structure. Past studies have observed that the combination of two or more mechanism changes the model output, compare to when you only analyze one, as well as the use of 2D or 3D scheme. The project would be a collaboration with Professor Shih-Nan Chen, which makes the project feasible in the time frame of the grant. His experience using numerical models to explain processes in coastal environments is a key component for the success of this work. This study will help develop scaling estimates that can be use to compare this region with others in the world, and help quantify the effect of this mechanism in the transport of heat and salt. 


\subsection{Collaboration with IONTU}

The ability to collect continuous data from the KC, has helped us understand the dynamics of this highly variable current that is vital for the balance of heat between the ocean and the atmosphere. IONTU scientific community has monitored the this current for decades, and has develop an extensive array of observations. The collective knowledge of this institution in the processes observed in the Kuroshio current, and the expertise in numerical modeling has motivated me to look for an opportunity to become part of this community. 

My experiences and background makes me an asset in observational labs such as IONTU, where the use of specialize equipment, like gliders, is increasing. In addition, this grant will give me the opportunity to share my experience with graduates and undergraduate students who start working with these instruments, and train them for successful field campaigns.



The collective knowledge of this institution in the processes observed in the Kuroshio current, and the expertise in numerical modeling has motivated me to look for an opportunity to work with this group, as well as the community in which they work and the colliage spirt that I have encountered in my visits.
IONTU has recently obtained two sea gliders, and this excite me to be part of the community that is also using this type of instrument as part of the research. As a direct result of this grant, I would like to help develop field campaigns and help IONTU collect data for future collaboration.

\subsubsection*{Working in Taiwan}
I have worked with different Taiwanese scholars as well as undergraduate and graduate students from various institutes on oceanographic cruises. I have experience with the research being done in this field in Taiwan waters and their operation.
I currently do not speak fluent in Mandarin, However, I have interacted constantly with students and scientists in English and the dissemination of the results and outcomes will be English. The faculty with wish I will be collaboration is fluent in English, as well as the other faculy in the IONTU and students too. The literature and conferecnes I will be attending is also in eglihs. I have address this issue with the faculty, and they do no see a problem. I have prevousely given a talk in english in the deparment in english and have been able to interact with students and grdaute students in englis. However I do se the value of speaking in the local language and will be taking classes of mandarng to be able to interact with the Taiwaneess community. 
I have ample experience workign with underwater vehciles, IONTU has recently aquries seagliders and I will be working with them to develope sampling palns and help them manage this instruments. 
To bettern udnestand how oceanography affects other communities, as in Taiwan which dpended in fisheries, I see as importat to give meaning to the field. I have work for my dissertation in this region, that I find in necesserary to understand more the community that has aidied this reseraech, and the reason on why is it important im sutyd int. So i se it as a natural progression to continue my work in TAiwan. 

Dissipation of the results for this studi will be done over publications and conferences, primarly I would like to attend seminar and lectures in Taiwan and other locationsin SOtheast asia do undergo resereach pertinant to the Kuroschio current and the southchina sea. Aslo  I would attend conferences in the united states to mainting contact with american collegues. 


\subsection{outcome}
This is important for me to develop better sampling technices que understand how the wya we sampel my influece the type of processe taht we can observec in the water ( here with you put a little bi tin the begginign my experience with intenxt campagning wiht glider) 

Another poing why this reserach in simporatn is to improve parametrization ( can mwe) of models and things like that. Our findings from the models experiemnts can be compared with data from the kuroshio current data setn IONTU has created and infomred how to plan other field campainges, 

Results of this study will be share in scientific journals and desiminated 

\section*{Research Philosophy}

\ As part of  my graduate educatin, I have also educate myself in issues of Social Justice, and have worked to tailored in my research philsphy alight with the principals and priorities that will create a more just academic community. I have collaborated with studets from other discipline to bring speakers within the univesity community to discuss isues of power dynamics in academia, race and higher education, equity in the field and lab and sexual harrasment. I have also taken courses in inclusive clasroom , and gener and health jsutice, which has given me a new perspective in the way science is done in the Unites states, and our global responsability to become a inclusive, equtily community. As a women with a latino background, I have xperiece first hand the obstacles faced to finish a degree is this field and have benefited from faculty from other disciplines, as ethcin studies and gender studies, that have helped find my voice and undestand how to navigate higher educaiotn. When Ilm a stablich sicitenc in my field, beause i will be no matter what, I also want to help others in completing their degree succesffully, in which ever whay they fell sucess is, an dempower them to do in in their own terms/ 
 I think that leaving abroad wil help me develop even more sensitiviy to this issues from the perspectia of an Asian Culture. 

This is an oppoertunty to experience and observe toher curlutes an d other sciences, and scientis, exploring the differences and not stay with the same community


I have had interact with undergradute and grad students from t and being thanksfull of creating connections that have helped me learn from them and viceverza. 

Immersing myself in a different culture ill help understand how to be supportive students from different backgroun

\end{document}