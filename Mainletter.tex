%%%%%%%%%%%%%%%%%%%%%%%%%%%%%%%%%%%%%%%%%%%%%%%%%%%%%%%%%%%%%%%%%%%%%
%% Title: SOP LaTeX Template
%% Author: Soonho Kong / soonhok@cs.cmu.edu
%% Created: 2012-11-12
%%%%%%%%%%%%%%%%%%%%%%%%%%%%%%%%%%%%%%%%%%%%%%%%%%%%%%%%%%%%%%%%%%%%%

%%%%%%%%%%%%%%%%%%%%%%%%%%%%%%%%%%%%%%%%%%%%%%%%%%%%%%%%%%%%%%%%%%%%%
%%
%% Requirement:
%%     You need to have the `Adobe Caslon Pro` font family.
%%     For more information, please visit:
%%     http://store1.adobe.com/cfusion/store/html/index.cfm?store=OLS-US&event=displayFontPackage&code=1712
%%
%% How to Compile:
%%     $ xelatex main.tex
%%
%%%%%%%%%%%%%%%%%%%%%%%%%%%%%%%%%%%%%%%%%%%%%%%%%%%%%%%%%%%%%%%%%%%%%

\documentclass[letterpaper, 12pt ]{article}
\usepackage[letterpaper,margin=1 in]{geometry}
\usepackage{fontspec, color, enumerate, sectsty}
\usepackage{indentfirst}
\setlength{\footskip}{20pt}
\sectionfont{\large}
\subsectionfont{\small}


\usepackage[normalem]{ulem}
%


%\usepackage{titlesec}
%\titleformat*{\section}{\fontsize{18}{20}\selectfont}
%\titleformat*{\subsection}{\fontsize{13}{17}\selectfont}
%%%%%%%%%%%%%%%%%%%%%%%%%%%%%%%%%%%%%%%%%%%%%%%%%%%%%%%%%%%%%%%%%%%%%
%                      YOUR INFORMATION
%
%      PLEASE EDIT THE FOLLOWING LINES ACCORDINGLY!!
%%%%%%%%%%%%%%%%%%%%%%%%%%%%%%%%%%%%%%%%%%%%%%%%%%%%%%%%%%%%%%%%%%%%%
\newcommand{\soptitle}{Fulbright Scholar Research Award Project Statement}
\newcommand{\yourname}{Alejandra Sanchez-Rios}
\newcommand{\youremail}{asanchez@ceoas.oregonstate.edu}

\usepackage[bookmarks, colorlinks, breaklinks,
pdftitle={\yourname - \soptitle},pdfauthor={\yourname}, unicode]{hyperref}
\hypersetup{linkcolor=magneta,citecolor=magenta,filecolor=magenta,urlcolor=[named]{WildStrawberry}}

\usepackage{setspace}
\pagestyle{plain}
\singlespacing

%%%%%%%%%%%%%%%%%%%%%%%%%%%%% HEADER

\usepackage{fancyhdr}
% \pagestyle{fancy}
% \fancyhf{}
% \rhead{Alejandra Sanchez-Rios}
% \lhead{Research Project Statement}

% \fancyhead[L]{\small Personal Statement}
%%%%%%%%%%%%%%%%%%%%%%%%%%%%% PAGE NUMBER

\usepackage{lastpage}
\rfoot{Page \thepage \ of \pageref{LastPage}}

%%%%%%%%%%%%%%%%%%%%%%%%%%%%%%%%%%%%%%%%%%%%%%%%%%%%%%%%%%%%%%%%%%%%%
%                      Title and Author Name
%%%%%%%%%%%%%%%%%%%%%%%%%%%%%%%%%%%%%%%%%%%%%%%%%%%%%%%%%%%%%%%%%%%%%
\begin{document}
%\thispagestyle{empty}
\begin{center}{\Large \scshape \soptitle}\end{center}
\begin{center}\vspace{0.2em} { \yourname\\}
\end{center}

%%%%%%%%%%%%%%%%%%%%%%%%%%%%%%%%%%%%%%%%%%%%%%%%%%%%%%%%%%%%%%%%%%%%%
%                      SOP Body
% NOTE: Use \amper instead of \&
%%%%%%%%%%%%%%%%%%%%%%%%%%%%%%%%%%%%%%%%%%%%%%%%%%%%%%%%%%%%%%%%%%%%%

Collaborative work has become an intrinsic strategy in the field of Physical Oceanography, and it has enabled us to study complex processes in different regions of the ocean. For instance, to analyze where mixing between different types of water masses occurs, field observations rely on constant surveillance with expensive and specialized equipment, e.g., microstructure sensors that measure turbulence. This economical and technical burden can be shared between different institutes and nations, allowing to plan a successful field campaign tailored to the nature of turbulence, a physical process that is highly intermittent and small in spatial and temporal scales. Projects also benefit from the collaboration between interdisciplinary scholars. A team composed of ocean modelers, observational oceanographers, physicists, atmospheric scientists, and engineers, can tackle a problem from different angles, advancing the field further and exploring new scientific questions. International collaboration creates an opportunity to study different regions of the ocean with similar processes, and creates a inclusive community that address global issues. As a PhD student at Oregon State University (OSU), I have been fortunate to work along side different institutes and different countries, specially in Taiwan, and I have come to understand how invaluable working in such a team is. As I transition to a new phase in my career, I want to continue building and strengthening connections with the Taiwanese scientific community, and The Fulbright scholar award would pave the way for me to collaborate with the Institute of Oceanography in the National Taiwan University (IONTU), but most importantly, will allow me to be part of a scientific community that I have only known as an external collaborator. The project I propose to carry out with Professor Shih-Nan Chen is the next step from my dissertation work; I plan to investigate and describe mechanisms that facilitate mixing of two different water masses at the edge of an ocean front. Through the collaboration with IONTU, I will have access to the extensive field data sets from the region and benefit from the experience of faculty using numerical simulations to model mixing mechanisms. Moreover, as a visiting scholar I can share my expertise in data analysis of small-scale features, and my field work experience with underwater vehicles and specialize equipment with the Taiwanese scientific community and IONTU. 


% I studied the Kuroshio Current, which  impacts atmospheric and ecological systems in Taiwan. 
\section*{Background \&  Research Experience}

% \vspace{10pt}
    % Before introducing the details and objectives of the proposed project,  processes that interest me and comment about current studies in the field that have lead to development of new questions and new challenges
    
% \subsection*{The Importance of Ocean Fronts}

\vspace{10pt}
    Ocean fronts are regions that bring together water masses of different characteristics. These differences creates gradients in temperature and salinity which are of particular interest due to the way both properties inversely affect density of seawater. Stable lateral gradients in density maintain ocean currents, and are reservoirs of potential energy. Observations and model outputs have shown how this equilibrium can be easily disturb by a wide type of factors: strong wind can inject momentum into the upper layers, heat exchange with the atmosphere will change the density of the water mass, temporal variability of the location of the front will change the density gradient, and differences in velocities at different depth will facilitate vertical shear, to name a couple of them. All these physical conditions facilitate processes that will enhance mixing between the two different water masses and are important because they enable exchange of heat, nutrients, inorganic and organic matter across different regions in the ocean and between the upper layers with deeper ocean.
    \vspace{10pt}
    
    The processes involve with mixing in frontal systems are diverse, a small list can include: instabilities in the mixed layer, buoyancy convection, frontogenesis, subduction of water mass, lateral and vertical shear, and many others. These processes often result in stirring at the edge of the front, forming small-scale features, e.g interleavings, filaments and eddies, in which mixing mechanisms can act upon faster. The study of these features has benefited by the advances in technology that permits fast and continues measurements, and the increase capabilities of computers to resolve high-resolution models. Research is being done to quantify the exchange of heat and salt in frontal regions, and understand which processes are the main players in this transport. Past studies have demonstrated that the interaction between the different process is also equally important. The study of mixing and how water density is changed, known as water transformation, is crucial for modeling how currents transport tracers across ocean basins, and for studying  how transport will be affected by global circulation and atmospheric changes. Ocean currents that transport great quantities of heat and salt across ocean basins are western boundary currents. My researched has focus on two of them: the Gulf Stream and the Kuroshio Current.
    
    
    % The air-sea interaction is also important to understand
    %  The upper ocean interacts with the atmosphere, sequestering CO2, exchanging heat, and acquiring O2. Water in the surface layers can then be transported into deeper depths, and deep water, rich in nutrients can be brought up to the surface where it can provide light and oxygen to the organism living in it. As its apeprent there are lot of processes that take place in ocean fronts, the complexity of the conectinos of these preocesses and ther sum impact is one fo the new frontie in oceangoraphy, In an effor toknow more about the fronts evulions, I have participated in crusies that have expliret he variability of thises proceses in kuroshio curent. This has inspired e to  know in detailed the processes that we have observed in the hight-resolution data obtained from the seagliders. 

 \subsection*{Previous research}  
    % I have extensive experience with oceanographic instruments that measure ocean properties needed to understand the dynamics occurring in the front, specially autonomous underwater vehicles, commonly knows as Gliders. I have been trained in deploying, piloting, and recovering them
    
    For my dissertation I have identified and characterized structures along ocean fronts that facilitate the lateral exchange of heat and salt across both currents: the Gulf Stream (GS) and the Kuroshio Current (KC). For the collection of data of these both studies, I was extensively involve in research cruises, and collaborated with faculty from different institutes in the US and Taiwan. In both cases we used satellite data to follow the location of the fronts, once identify, with deployed several instruments and profile the region. We used data collected from gliders, microstructure sensors and fast profiling conductivity, depth and salinity profilers to characterized the front. We carry out three cruises south of Taiwan in collaboration with faculty from IONTU. During this field work, I was trained in the deployment and recovery of oceanographic instruments, and as part of OSU glider team, I helped set up, pilot, and retreive data from gliders. 
    \vspace{10pt}
    I invested a lot time in the analysis of the extensive and complex data set collected in both regions. Using the high-resolution \textit{in situ} measurements and satellite data, we created a 3D image (30 km across, 600 km along and 200m deep) of the Gulf Stream front over four days and evaluated the change in time of both heat and salt in the upper layer. We identified several small-scale features (10-50 km) of warm water being carry across the front and observed there was an increase of temperature and salinity associated to the location of the front. We concluded that the only possible source for the change, was lateral exchange, and the estimated rate was similar to past studies values. I also investigated  water transformation at the edge of the Kuroshio Current as it intrudes to the South China Sea. We identify temperature and salinity interleaving through the water column at the edge of the KC, and explored the mechanisms responsible for mixing; double diffusion and shear turbulence int his case. We found that both mechanism were present in the data set, and the impact one has over the other is still being investigated. This project is part of a Office of Naval Research (ONR) partnership with IONTU, were several US institutes participate including OSU, being involve with this partnership has permitted me to understand how both parties benefit from this team effort: research done in the region, from both Taiwanese scholars and US scholars, is important for the local community as it impacts fisheries, ecology and weather patterns, and for the global characterization of ocean process that regulate the transport of heat and salt. This partnership grants IONTU access to oceanographic instruments from US institution, and the data collected, has increase our understanding of frontal regions such as the ones form by the KC. As I worked in the analysis of data collected in the Souch China Sea, I was able to share my results to IONTU faculty, and improve my work from their advice and comments, the collection of the data set would not been possible if it was not for the collaboration with Taiwanese scholars. I want to learned more from the community that has aided me in my research of the KC, strengthening the connection.
    

    % maybe mention in this part how this partnership allowed me to learn the research projects carry out in IONTU, and have a glimpse of the academic cultural 
    
% and correlated this features with the observed changes.  labeled streamers. Streamers are warm salty water  from the Gulf stream interleaved with the cold-fresh water of the Slop Water from the Mid-Atlantic bit. We concluded that thes
    % Observting the complexity and the diversty of procesies in has inspired me to dig deeper in understanding them. And Interacting with several resereacher from the IONTU in conferences and report meetings I have learned the resources and the collective exprienece these institution has about this area and experience owrking with numerical models which help them study with more detail thier thoeries and
    
% \subsection*{The Kuroshio Current}
% \begin{enumerate}
% \vspace{10pt}
%     \item  O
    
%     \item Description of the Kuroshio Current and the intrusions into South China Sea
    
%     \item Description of studies done east of the Kuroshio current and in the South China Sea
    
%     % This currents is born when the west bound equatorian current meet the Philippians and splits in two, the MC which travels south, and the Kuroshio current that travels north. Once in its way towards the north, the kuroshio current will continue and pass east of Taiwan and and hit the East China sea, where it will then be deflectes towards the east. The Kuroshio current can also intrude thrught the luzon straing towards the South China Sea, concting this basin with the Pacific ocean. The nature of this intrusin is vastly study, and the proceses that involve the mixing of the water transported by th eKC and the SCS water was the focus of the second chapter of my dissertaion. 
    
%     \end{enumerate}
\section*{Project Overview}
% 
One of the main reasons I want to carry out a research project in Taiwan, is the direct impact the Kuroshio current has over this region. The KC impacts storms tracks, regional climate and the marine ecosystem. From the data collected in the KC front, I was able to observed a huge range of density, temperature and salinity structures and patterns, which has inspired me to investigate further how these are form in the front and dive deeper into the physical theory behind them, what process are more important? How do they relate to regional circulation? How common are in the region? What triggers them?. During my time in IONTU, I propose to use numerical models to investigate different mechanisms responsible for mixing at ocean fronts that enables vertical and lateral transport along the front, and compare them to real cases. 

 \subsection*{Objectives \& Methodology}
 
The main objective would be to explore the effect of wind forcing, instabilities, and stratification in the density structure of a front similar to the ones KC forms, and investigate how this effects lead to water mass transformation. As well, to explore the type of patterns develop in the front when these mechanism are present, and compare it to past observations of the KC. Using  idealized numerical simulations, will model realistic scenarios known to destabilized fronts,  e.g., downfront wind and baroclinic instability, to analyze their effect in the density and velocity three-dimensional structure. Past studies have observed that the interaction between the mechanisms is equally important as there individual presence, as the combination of two or more mechanism changes the results. The project would be a collaboration with Professor Shih-Nan Chen, which makes the project feasible in the time frame of the grant. His experience using numerical models to explain processes in coastal environments is a key component for the success of this work. This study will help develop scaling estimates that can be use to compare this region with others in the world, and help quantify the effect of this mechanisms in the transport of heat and salt. 

\subsection*{Collaboration with IONTU}

The ability to collect continuous data from the KC, has helped us understand the dynamics of this highly variable current that is vital for the balance of heat between the ocean and the atmosphere. IONTU scientific community has monitored this current for decades, and has develop an extensive array of observations. The collective knowledge of this institution of the processes observed in the KC, and their expertise in numerical modeling has motivated me to look for an opportunity to become part of this community. IONTU has recently acquired two sea gliders, and my experience with this type of specialize equipment make me an asset to this group. This grant will give me the opportunity to share my experience with graduates and undergraduate students who start working with these instruments, and train them for successful field campaigns.
\vspace{10pt}
I have worked along side different Taiwanese scholars as well as undergraduate and graduate students from various Taiwanese institutes on oceanographic cruises and I have observed the type of research carry out in IONTU through meetings held in Taipei and field work, where I also was able to share my research in the KC. I'm currently not fluent in Mandarin, however, the faculty and students in IONTU are fluent in English and I have been able to interact constantly with them during my fieldwork and visits: I presented a talk in spring and was able to engage in discussions with the faculty. The dissemination of the results and outcomes will also be in English. I have address this issue of not speaking fluent Mandarin with the faculty, and they do no see it as a problem. Regardless of currently being able to communicate with the IONTU faculty in English, I see the value of speaking in the local language and will be taking language classes during the coming year to be able to interact with the Taiwanese community.

% The collective knowledge of this institution in the processes observed in the Kuroshio current, and the expertise in numerical modeling has motivated me to look for an opportunity to work with this group, as well as the community in which they work and the colliage spirt that I have encountered in my visits.
% IONTU has recently obtained two sea gliders, and this excite me to be part of the community that is also using this type of instrument as part of the research. As a direct result of this grant, I would like to help develop field campaigns and help IONTU collect data for future collaboration.

% \subsubsection*{Working in Taiwan}

\subsection*{Outcome}

The results from this project will provide mixing estimates for frontal regions linked to specific process. As the numerical simulation will be based out from realistic conditions, we will relate the model output with past data from the Kuroshio Current. Understanding how specific mechanism affect mixing in the KC, and how the combination of them impact the end result is also import for the planning and development of future survey campaigns. The more we know about the time and spatial scales of the process we are interested in, the better we can prepare field work aimed to survey them, and the more we can improve our current instruments. Estimates of transport and the study of mixing rates are important for the correct parameterization of mixing in regional and global circulation models. Results from our project will be compare with current mixing parametrizaion schemes for the KC system, and analyze to understand discrepancies or similarities.  Results of this study will be share with the scientific community via manuscripts in journals and disseminated in international and US conference 

% (presence in the Internet )

\section*{Research Philosophy \& Motivation}
% (small sentence of  introduction)
As I worked on my research, I have educated myself in issues of Social Justice, by taking part of workshops and classes that promote inclusive teaching pedagogues, and discuss topics of equity in higher education. I have collaborated with students from other disciplines to bring speakers within the university community to talk about of power dynamics in academia, race in higher education, equity in the field and sexual harassment. This group brings together faculty and graduate students to start conversations that are in times difficult to have, but build a strong community. These classes and discussions have given me a new perspective on the way we teach and do science in the Unites States academia: it has open my eyes to the our global responsibility to become an more inclusive and equitable community. To create a work-space that is truly diverse by not just welcoming different ways of thinking, but valuing equally different ways of thinking and different personalities. I also believe in the importance of working with people from other disciplines to strengthen our oceanographic community. As a women with a Latino background, I have benefited from the mentoring of faculty from other disciplines, as ethnic studies and gender studies, that have help me find my unique voice in academia,  successfully  navigating around obstacles in higher education, and helped me establish meaningful connections with our local community. I'm actively involve with mentoring undergraduate and graduate students from latino background, and volunteer in committees that aim in advancing minorities groups in STEM fields.  I'm committed to build an inclusive and welcoming environment in which ever institute, lab and field work I participate in, I want to empower students and faculty from different backgrounds to embrace their own unique voice in the field, and help, as many have help me, continue working on physical oceanography

\vspace{10pt}
The opportunity to do research and live in Taiwan, is invaluable to me. The propose research project will give me a new set of tools for investigating the dynamics of processes I have been sampling and describing for years, while collaborating with faculty and students from a different background as mine. I'll be able to become part of a different community, with different culture and language, while working with leaders in my field. This experience will help me create a deeper collaboration with academic circles outside from United States, continuing the work of others in creating a global web of researchers. I believe that this type of partnerships are the way of solving the future challenges in  oceanography, challenges that directly impact our global climate and oceans. 

 



\end{document}