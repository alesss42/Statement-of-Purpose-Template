%%%%%%%%%%%%%%%%%%%%%%%%%%%%%%%%%%%%%%%%%%%%%%%%%%%%%%%%%%%%%%%%%%%%%
%% Title: SOP LaTeX Template
%% Author: Soonho Kong / soonhok@cs.cmu.edu
%% Created: 2012-11-12
%%%%%%%%%%%%%%%%%%%%%%%%%%%%%%%%%%%%%%%%%%%%%%%%%%%%%%%%%%%%%%%%%%%%%

%%%%%%%%%%%%%%%%%%%%%%%%%%%%%%%%%%%%%%%%%%%%%%%%%%%%%%%%%%%%%%%%%%%%%
%%
%% Requirement:
%%     You need to have the `Adobe Caslon Pro` font family.
%%     For more information, please visit:
%%     http://store1.adobe.com/cfusion/store/html/index.cfm?store=OLS-US&event=displayFontPackage&code=1712
%%
%% How to Compile:
%%     $ xelatex main.tex
%%
%%%%%%%%%%%%%%%%%%%%%%%%%%%%%%%%%%%%%%%%%%%%%%%%%%%%%%%%%%%%%%%%%%%%%

\documentclass[letterpaper, 12pt ]{article}
\usepackage[letterpaper,margin=1 in,noheadfoot]{geometry}
\usepackage{fontspec, color, enumerate, sectsty}
\sectionfont{\large}
\subsectionfont{\normalsize}


\usepackage[normalem]{ulem}
%


%\usepackage{titlesec}
%\titleformat*{\section}{\fontsize{18}{20}\selectfont}
%\titleformat*{\subsection}{\fontsize{13}{17}\selectfont}
%%%%%%%%%%%%%%%%%%%%%%%%%%%%%%%%%%%%%%%%%%%%%%%%%%%%%%%%%%%%%%%%%%%%%
%                      YOUR INFORMATION
%
%      PLEASE EDIT THE FOLLOWING LINES ACCORDINGLY!!
%%%%%%%%%%%%%%%%%%%%%%%%%%%%%%%%%%%%%%%%%%%%%%%%%%%%%%%%%%%%%%%%%%%%%
\newcommand{\soptitle}{Fulbright Scholar Research Award Project Statement}
\newcommand{\yourname}{Alejandra Sanchez-Rios}
\newcommand{\youremail}{asanchez@ceoas.oregonstate.edu}

\usepackage[bookmarks, colorlinks, breaklinks,
pdftitle={\yourname - \soptitle},pdfauthor={\yourname}, unicode]{hyperref}
\hypersetup{linkcolor=magneta,citecolor=magenta,filecolor=magenta,urlcolor=[named]{WildStrawberry}}

\usepackage{setspace}
\pagestyle{plain}
\singlespacing
%%%%%%%%%%%%%%%%%%%%%%%%%%%%%%%%%%%%%%%%%%%%%%%%%%%%%%%%%%%%%%%%%%%%%
%                      Title and Author Name
%%%%%%%%%%%%%%%%%%%%%%%%%%%%%%%%%%%%%%%%%%%%%%%%%%%%%%%%%%%%%%%%%%%%%
\begin{document}
\begin{center}{\Large \scshape \soptitle}\end{center}
\begin{center}\vspace{0.2em} { \yourname\\}
\end{center}

%%%%%%%%%%%%%%%%%%%%%%%%%%%%%%%%%%%%%%%%%%%%%%%%%%%%%%%%%%%%%%%%%%%%%
%                      SOP Body
% NOTE: Use \amper instead of \&
%%%%%%%%%%%%%%%%%%%%%%%%%%%%%%%%%%%%%%%%%%%%%%%%%%%%%%%%%%%%%%%%%%%%%

Collaborative work has become an intrinsic strategy in the field of Physical Oceanography to resolve current scientific questions and to enable us to study complex processes in different regions of the ocean that were not possible before. For instance, to analyze where mixing occurs in the ocean at small-scales (<10 km), field observations rely on constant surveillance with expensive and specialized equipment, e.g., microstructure sensors to measure turbulence and thermal and salinity variability. This economical burden can be shared between different institutes and nations, bringing together sufficient equipment to create a successful campaign tailored to the nature of turbulence, a physical process that is highly intermittent and small in spacial and temporal scales. This way, projects also benefit from the collaboration between interdisciplinary scholars. A team composed of ocean modelers, observational oceanographers, physicist, atmospheric scientist and engineers, can tackle a problem from different angles, advancing the science further and exploring new scientific questions. Equally important, is the collaboration with fields such as policy making, ecology and ......, which help steer a project objective and put the results into a socio-economic context. As a Ph.D. at Oregon State University (OSU), I have been fortunate to work along side different institute and different countries (describe in what project), and I have come to understand how invaluable working in such environments is. As I transition to a new phase in my career, I want to continue building and strengthening connections with the Taiwanese scientific community. The Fulbright scholar award can permit me to collaborate with the  Institute of Oceanography in the National Taiwan University (IONTU), but most importantly, will allow me to be part of a scientific community that I have only known as an external collaborator. The project I propose to carry on with collaboration with faculty member Dr, Shih-NAn Chen, I the next step from my dissertation work, where I looked at different ocean fronts and investigated how important these regions are for global estimates of heat and salt transport. Specifically, I studied the Kuroshio Current, which  impacts atmospheric and ecological systems in Taiwan. Through the collaboration with IONTU, I could have access to the extensive field data sets they have been able to collect and to theier experience with numercial simulation of the dynamics of this region. As a visiting scholar I can share my expertise in data analysis and managing of autonomous surveillance vehicles as the glider. 


\subsection*{The Importance of Ocean Fronts}

\begin{enumerate}
    \item Ocean fronts are regions in the ocean that bring together water masses of different characteristics. This differences creates gradients in temperature and salinity principal interest are front created by buoyancy gradients, common in western boundary currents, which are warm, salty, and light water mass, when they encounter cold, fresh, and dense water. Ocean fronts   
    \item The upper ocean interacts with the atmosphere, sequestering CO2, exchanging heat with the atmosphere, it sets the level of oxygen. Water in the surface with set characteist can then be transported into deeper depths, and deep water, rich in nutrients can be brought up to the surface where it can provide light and oxygen to the organism living in it. As its apeprent there are lot of processes that take place in ocean fronts, the complexity of the conectinos of these preocesses and ther sum impact is one fo the new frontie in oceangoraphy, In an effor toknow more about the fronts evulions, I have participated in crusies that have expliret he variability of thises proceses in kuroshio curent. This has inspired e to  know in detailed the processes that we have observed in the hight-resolution data obtained from the seagliders. 
Observting the complexity and the diversty of procesies in has inspired me to dig deeper in understanding them. And Interacting with several resereacher from the IONTU in conferences and report meetings I have learned the resources and the collective exprienece these institution has about this area and experience owrking with numerical models which help them study with more detail thier thoeries and



    \item Related to our study is the Kuroshio currents, wich carry North Pacific tropical water towards the norths. The KC plays a vital role in the transpor of heat from the equator towards the polar region. 
    \item 
\end{enumerate}

\subsection*{Background \&  Research Experience}
\begin{enumerate}
    \item The main focus of my research has been the study of structures along ocean fronts, and the study of the mixing processes that are related to them that affect the heat and salt budgets of the upper ocean. 
    
    
    \item I have been involve in research cruises extensively, and collaborated with faculty from different institutes in the US. For the study of small-scale features in the Gulf Stream north wall the collaboration team shares data from different instruments to put together a 3D images of the GS front.  and study the heat and salt content evolution of the upper layer of the ocean. We follow the density field along the front and analyzed the impact small scale interleaving of cold and warm water has in the lateral transport of both heat and salt, with the goal to better estimate the role lateral mixing has in the dissipation of tracers and energy. This work was carry out in collaboration with faculty in different institutes in the US. 
    
    \item For may dissertation I investigated  water transformation at the edge of the Kuroshio branch current as it intrudes the South China sea and mixes. I participated in three month-long cruises off the coast of Taiwan, which were in collaboration with faculty from IONTU,surveying the front the KC forms with the cold water of the South China sea  wit the goal to first, desciribe the type of processes and to estimate the mixing coefficent in this region when this thpe of current occue. The heat transport from the kurosho curernt towards the SCS trougth luzon streing its needed to underatnt the the interatino betwen the Paciic ocean ant the South chiana sea. 
    
    \item Summary some of the maing conclusions of the work I have done
\end{enumerate}


\section*{Project Overview}
During my tain in IONTU we wills tudy speciif dynamicla process in ocean fronts that enchanges mixing, with stess on the regions with strong lateral gradietns. Using models we will invitgage the processes in the front that lead to mixing and comprehend the mechanism that enables them. Previous research have shown how individual process have a dfferent effect inthe upper mixed layer compared to when taken into account with other process. ( examle the paper we read). 
\subsection*{Objectives}

We will investigage the effec of barocliniciy, wind forcing and other procesed that we can decid when we observe more data. The main objective will be to investigabe the eedback between different mechanism and try to better undersatand how complex process intercat with eachtoher to enable observeve mxing and dissiptions values. 


\subsection*{Methodology}
Aided by the lage data set IONTU has of observations of the Kuroshio current, we will selected a region and front characterist to investigage. 

The proposed research project entails the use of idealize numerical simulations to study the processes that affect lateral and vertical mixing in regions with large buoyancy gradients  gradients, e.g. downfront wind, baroclinic  instability,  by the formation of interleaving features that are commonly seen in ocean fronts, like the one Kuroshio currents forms eighet when in Intrudes into the South China Sea through the Luzon strait or east of Taiwan with the cold water from the East China Sea.
For the creation of the numerical model, data from large data base IONTU has from the Kuroshio current will be use to realistic case. This will help develop scaling estimates that can be use to compare with other regions. 
Analysis of past observations in both region will help informed the the attributes of the numerical model. 
Compare the resoults of the models with observation data, and describe the different processes that are affecting the mixing regime in the Kuroshio current. 


\subsection*{Collaboration with IONTU}

The experience of the  porporse collaborator, Dr. Shih-Nan Chen,  makes the project feasible in the time of the gran,  his past work with simulations is crucial for the succes of this work, he has also used model to udnerstand proces in coastal environemnet,  this well aid with the set up of the model, and then continue with the experiment where will evluate differetn cases

I have experience with oceanographic instruments that measure ocean propoerties needed to understand the dynamics occuring in the front, specially autonomous underwater vehiles, comminly knows as Gliders, I have een trained in piloting them and deploy and recover them. This makes me an asses in observational labs where the use of this gliders is increasin just likt IONTU. Where I can shere my experience to graduates students who start working with this instruemtn,s and help planed sucesfull vield campaintes

The ability to collect data, and work with data from the Kuroshio current, east of tAiwan and South of Taiwan, has helps us understand the dunamics of a highly variable current  that is vital for the balance of heat between the work ocean and the atmosphers. In contraru with the Gulf stream wich alo carry warm water from the subtropcial water toward the poles,Kuroshio current have dfiferent patha of intrusion int o a nother basin, e.g. South china SEa, and meanders extenseve tis proximinty to the Taiwan coast.

The collective knolesge of this instituion in the proceses observed in the Kuroshio curent, and the experietsi in numerical modeling has motivate me to look for opportunity to work with this group, as well as the community in whih they work and the colliage spirt that I have encoutner when I have visit to give  talk. 
They have recently obtained two seagliders, and excite me to be part of the community that is also using this type of instrument, as part of the research I would like to help develop field campgines and help them collect data for future collaboration. 
\subsubsection{Working in Taiwan}
I have work with Taiwaneses from different institutes in oceanographic cruises, and also undergraduate and graduate students from this institutes. I have experience with the research being done in this field in Taiwanes waters and there operation.
I currently do not speak fluent in Mandarin, However, I have interact constatnly with studentds and scientines in english and the desiminaion of the results and outcomes will be English. The faculty with wish I will be collaboration is fluent in English, as well as the other faculy in the IONTU and students too. The literature and conferecnes I will be attending is also in eglihs. I have address this issue with the faculty, and they do no see a problem. I have prevousely given a talk in english in the deparment in english and have been able to interact with students and grdaute students in englis. However I do se the value of speaking in the local language and will be taking classes of mandarng to be able to interact with the Taiwaneess community. 
I have ample experience workign with underwater vehciles, IONTU has recently aquries seagliders and I will be working with them to develope sampling palns and help them manage this instruments. 
To bettern udnestand how oceanography affects other communities, as in Taiwan which dpended in fisheries, I see as importat to give meaning to the field. I have work for my dissertation in this region, that I find in necesserary to understand more the community that has aidied this reseraech, and the reason on why is it important im sutyd int. So i se it as a natural progression to continue my work in TAiwan. 

Dissipation of the results for this studi will be done over publications and conferences, primarly I would like to attend seminar and lectures in Taiwan and other locationsin SOtheast asia do undergo resereach pertinant to the Kuroschio current and the southchina sea. Aslo  I would attend conferences in the united states to mainting contact with american collegues. 


\subsection{outcome}
This is important for me to develop better sampling technices que understand how the wya we sampel my influece the type of processe taht we can observec in the water ( here with you put a little bi tin the begginign my experience with intenxt campagning wiht glider) 

Another poing why this reserach in simporatn is to improve parametrization ( can mwe) of models and things like that. Our findings from the models experiemnts can be compared with data from the kuroshio current data setn IONTU has created and infomred how to plan other field campainges, 

Results of this study will be share in scientific journals and desiminated 

\section*{Research Philosophy}

\ As part of  my graduate educatin, I have also educate myself in issues of Social Justice, and have worked to tailored in my research philsphy alight with the principals and priorities that will create a more just academic community. I have collaborated with studets from other discipline to bring speakers within the univesity community to discuss isues of power dynamics in academia, race and higher education, equity in the field and lab and sexual harrasment. I have also taken courses in inclusive clasroom , and gener and health jsutice, which has given me a new perspective in the way science is done in the Unites states, and our global responsability to become a inclusive, equtily community. As a women with a latino background, I have xperiece first hand the obstacles faced to finish a degree is this field and have benefited from faculty from other disciplines, as ethcin studies and gender studies, that have helped find my voice and undestand how to navigate higher educaiotn. When Ilm a stablich sicitenc in my field, beause i will be no matter what, I also want to help others in completing their degree succesffully, in which ever whay they fell sucess is, an dempower them to do in in their own terms/ 
 I think that leaving abroad wil help me develop even more sensitiviy to this issues from the perspectia of an Asian Culture. 

This is an oppoertunty to experience and observe toher curlutes an d other sciences, and scientis, exploring the differences and not stay with the same community


I have had interact with undergradute and grad students from t and being thanksfull of creating connections that have helped me learn from them and viceverza. 

Immersing myself in a different culture ill help understand how to be supportive students from different backgroun

\end{document}